\chapter{Metodologia}
O primeiro passo para a condução deste trabalho foi a verificação da existência de sistemas para identificação de chamadas falsas dentre a literatura acadêmica. Com esse objetivo, diversas consultas em repositórios de artigos e publicações foram feitas, tais como \textit{Google Scholar}, Portal de periódicos da CAPES, \textit{ACM Digital Library} e \textit{dblp computer science bibliography}. Utilizando como índices, palavras chaves como: \textit{emergency calls, prank calls, hoax calls, emergency service, 911 calls, 911 emergency calls, automatic prank call rejection}. Contudo, os resultados não foram satisfatórios. Ao menos inicialmente, acredita-se que não exista algum sistema de detecção de chamados falsos baseado no conteúdo da chamada, que esteja disponível. Logo, como sua viabilidade ainda é uma incógnita, este trabalho têm como objetivo avaliar a viabilidade de elaboração de sistemas que identifiquem corretamente quando uma chamada emergencial é falsa, com base no conteúdo da ligação, e por meio de classificadores.

Para alcançar esse objetivo, é necessário a execução de passos sequenciais, incrementalmente, ordenados a seguir: 
\begin{enumerate}
    \item Obter, analisar e compreender a base de dados. Como também, levantar hipóteses iniciais e definir um truncamento do período em que será analisado.
    \item Identificar, selecionar e extrair características, para facilitar a classificação. Como também a criação do \textit{target} que deseja-se prever.
    \item Treinar os algoritmos de classificação de texto, como RNA e KNN, definindo seus parâmetros empiricamente.
    \item Executar os algoritmos categorizadores e avaliar a predição fornecida por cada classificador.
    \item Escrever, concluir e apresentar os resultados obtidos.
\end{enumerate}

Ao fim, com os resultados obtidos, será possível mensurar preliminarmente se é viável tal sistema. Apresentando as estatísticas obtidas, as características identificadas, a performance individual de cada classificador utilizado e a análise crítica dos resultados. É importante ressaltar também, que caso o estudo não conclua positivamente, não significa que não exista viabilidade de tal sistema e sim que seja inviável nas circunstâncias em que foram testadas, com as especifidades dos dados, o idioma, características e métodos.

\section{Ferramentas de apoio}
Existem diversas ferramentas de apoio disponíveis que serão utilizadas de maneira intermediária para análise da viabilidade desse sistema. 

A principal delas é a linguagem de programação Python, criada por \cite{python}, que por ser uma linguagem estável, flexível e possuir um grande leque de ferramentas gratuitas disponíveis, vem se destacando em projetos de inteligência artificial e aprendizado de máquina. Dentre tais ferramentas, algumas serão aplicadas neste projeto, como \textit{scikit-learn} \citep{scikitlearn}, \textit{pandas} \citep{pandas}, \textit{NumPy} \citep{van2011numpy}, \textit{Natural Language Toolkit} (NLTK) \citep{nltk} e \textit{matplotlib} \citep{matplotlib}.

A biblioteca \textit{scikit-learn} dispõem de diversas funcionalidades que irão facilitar a implementação dos algoritmos de classificação e execução dos testes. Por exemplo, ela fornece a implementação da RN supervisionada com topologia FNN. Com os métodos disponíveis, caberá ao trabalho uma análise aprofundada dos parâmetros, como também apresentar uma validação cruzada, que significa avaliar a capacidade de generalização do modelo proposto, a partir de um conjunto de dados, e a análise dos resultados.

A biblioteca \textit{pandas} fornece ferramentas de análise de dados, estruturas de dados de alta performance e também leitura de arquivos, inclusive no formato \textit{Comma Separated Values - CSV} em que se encontra a base de dados planejada.

\textit{NumPy} é um pacote para a linguagem Python que suporta \textit{arrays} e matrizes multidimensionais, possuindo uma larga coleção de funções matemáticas para trabalhar com estas estruturas. Facilitando assim, as manipulações algébricas e cálculos científicos que venha a ser necessários.

A biblioteca NLTK é um recurso que dispõem diversos métodos para facilitar a manipulação da linguagem natural, como \textit{tokenization, stemming, tagging, parsing, semantic reasonning} e \textit{wrappers}. Vindo a ser muito útil durante a etapa de normalização dos dados.

Por fim, a biblioteca \textit{matplotlib} que é uma biblioteca de plotagem e montagem de gráficos para o Python e sua extensão matemática numérica \textit{NumPy}. Ele fornece uma API orientada a objetos para incorporar gráficos em aplicativos usando \textit{kits} de ferramentas \textit{Graphical User Interface - GUI} para uso geral.

\section{Base de dados}
Devido a dificuldade de encontrar uma base especifica para o problema, uma base que será adaptada é \textit{Call Data}, do centro de comunicações do departamento de polícia de Seattle (\textit{Seattle Police Department Communications Center} - SPD), dos Estados Unidos da América. Esta base de dados representa os relatórios das chamadas emergenciais originados pela comunidade local, na qual é armazenado cerca de $95\%$ de todos os chamados recebidos pelo departamento. No total, estão disponíveis 11 colunas de informações que descrevem a ocorrência (Tabela \ref{tab:callDataColumns}). Existindo mais de 3,9 milhões ocorrências únicas registradas desde 6 de fevereiro de 2009. Novos dados são inseridos diariamente pela equipe administradora, o que é um ponto positivo, pois é possível estimar o número de chamadas falsas recebidas em relação ao número total de chamados nos últimos anos, como também validar os objetivos ressaltados neste trabalho no cenário atual.

\begin{table}[ht!]
\caption{Tabela de atributos que descrevem uma ocorrência da base de dados \textit{Call Data}.}
\label{tab:callDataColumns}
\centering
\resizebox{\textwidth}{!}{%
\begin{tabular}{@{}llc@{}}
\toprule
\multicolumn{1}{c}{\textbf{Atributo}} & \multicolumn{1}{c}{\textbf{Descrição}}                           & \textbf{Formato} \\ \midrule
ID                                    & Identificador único                                              & Texto         \\ \midrule
Registro                              & Como foi resolvido o chamado                                     & Texto         \\ \midrule
Chamada                       & Origem da chamada (telefone, 911, alarme)                                               & Texto         \\ \midrule
Prioridade                            & Prioridade assimilada ao chamado                                 & Texto         \\ \midrule
Classificação inicial                 & Como foi classificado inicialmente a ocorrência                  & Texto         \\ \midrule
Classificação final                   & Como foi classificada pós o acompanhamento                         & Texto         \\ \midrule
Data e hora                           & Data e hora em que foi recebida o chamado                        & Data e hora   \\ \midrule
Tempo de chegada                      & Tempo de chegada do primeiro policial ao local do registro & Texto         \\ \midrule
Ponto cardeal                         & Ponto cardeal em relação ao mapa geográfico da cidade            & Texto         \\ \midrule
Setor                                 & Setor do ponto cardeal da ocorrência                             & Texto         \\ \midrule
Quadra                                & Quadra do setor da ocorrência                                    & Texto         \\ \bottomrule
\end{tabular}%
}
\end{table}

Por se tratar de uma base de dados pública e disponível na internet, as informações disponíveis são restritas em relação aos dados originais que são coletados diariamente pelo departamento, pois devido a sua natureza, são informações sigilosas e privadas da comunidade local. Dados como: nome, idade, sexo ou logradouro da ocorrência, foram removidos pelos administradores com o objetivo de preservar a privacidade dos envolvidos. 

Outra característica dessa base de dados é que ela está disponível no idioma inglês e com informações e relatórios dos chamados já bem resumidos, o que não possibilita tanta exploração e uso de técnicas de processamento de linguagem natural. Contudo, como toda base de dados, um pré-processamento é necessário, com objetivo de eliminar ou corrigir erros de ortografia, ambiguidades e informações irrelevantes.

\section{Cronograma de atividades}
O cronograma de atividades planejadas para execução deste projeto foi definido para os próximos cinco meses, começando a partir de agosto até dezembro de 2019. As atividades foram definidas e separadas por etapas (Tabela \ref{tab:atividades}), sendo elas:
    
\begin{enumerate}
    \item Análises sobre a base de dados.
    \item Implementação dos algoritmos e testes.
    \item Validação de resultados e discussão.
    \item Escrita, conclusão e apresentação dos resultados obtidos.
\end{enumerate}

\begin{table}[ht!]
\caption{Cronograma de atividades planejadas para a segunda etapa deste projeto.}
\label{tab:atividades}
\centering
%\resizebox{\textwidth}{!}{%
\begin{tabular}{cclllc}
\hline
\multicolumn{1}{l}{Mês/Etapa} & \multicolumn{1}{l}{{\color[HTML]{333333} Agosto}} & Setembro & Outubro & Novembro & \multicolumn{1}{l}{Dezembro} \\ \hline
\multicolumn{1}{c|}{1} & \multicolumn{1}{c|}{\cellcolor[HTML]{BCFFBC}} & \multicolumn{1}{c}{} & \multicolumn{1}{c}{} & \multicolumn{1}{c}{} &  \\ \hline
\multicolumn{1}{c|}{2} & \multicolumn{3}{c|}{\cellcolor[HTML]{FFFC9E}} & \multicolumn{1}{c}{} &  \\ \hline
3 & \multicolumn{1}{c|}{{\color[HTML]{333333} }} & \multicolumn{3}{c|}{\cellcolor[HTML]{FFCE93}} &  \\ \hline
4 & \multicolumn{1}{l}{} &  &  & \multicolumn{1}{l|}{} & \multicolumn{1}{l|}{\cellcolor[HTML]{FFADA9}} \\ \hline
\end{tabular}%
\end{table}