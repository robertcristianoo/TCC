Desde a efetivação dos serviços de urgência no Brasil, o número de chamadas falsas ou trotes recebidos pelos serviços de emergência aumentou drasticamente. Consumindo indevidamente os recursos públicos, gerando transtornos e custos extras a sociedade. Por este fato, surge a necessidade de ferramentas computacionais que auxiliem tais serviços, facilitando a identificação de chamados falsos, e evitando assim, a mobilização indevida desses recursos. Portanto, o objetivo deste trabalho é avaliar a viabilidade de sistemas que sejam capazes de identificar corretamente chamadas falsas, com base no conteúdo da ligação, através do auxílio de técnicas de aprendizado de máquina, tais como processamento de linguagem natural e classificadores de texto.

\keywords{Chamadas falsas, chamadas de emergência, classificação de texto, redes neurais, processamento de linguagem natural}