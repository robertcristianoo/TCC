Since the implementation of emergency services in Brazil, the number of fake or prank calls received by emergency services has increased dramatically. Consequently, public resources have been constantly wasted, generating problems and inneficience. Due to this fact, is lacking computational tools to support such services, that could be used to facilitate the correct identification of those fake calls, and thus avoiding undue mobilization of these resources. Therefore, this work propose evaluation of the feasibility of systems that are able to correctly identify fake calls, basedly on the content of the call, through the aid of machine learning algorithms, such as natural language processing and text classifiers.

\keywords{Emergency calls, prank calls, text classification, neural networks, natural language processing}