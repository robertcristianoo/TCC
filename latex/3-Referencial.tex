\chapter{Referencial Teórico}
Neste capítulo, será apresentado trabalhos e pesquisas existentes na área que visão solucionar parcialmente ou completamente o problema de identificação de chamadas emergenciais falsas. Um trabalho soluciona parcialmente o problema, se o mesmo dispor alguma metodologia útil para resolver o problema geral. Como por exemplo técnicas de classificação de dados eficientes para texto, ou algum processamento de linguagem natural que facilite a seleção das características relevantes. Primeiramente apresenta-se ...

\section{Classificadores de texto}
Uma etapa importante durante o processo de identificação de chamadas falsas é a obtenção do conteúdo da chamada em formato de texto. Após esse passo e com o conteúdo da chamada em formato de texto, é possível a aplicação de métodos de extração e seleção de características relevantes para então classificar-lo em uma categoria previamente conhecida.

Algoritmos de classificação de texto vêm sidos aplicados com sucesso em vários diferentes  problemas de diversas áreas de atuação e interesse, tais como rotulação de conteúdos, produtos, multimédia; Gerenciamento de Relacionamento com o Cliente (do inglês: \textit{Customer Relationship Management} - CRM) para empresas; monitoramento de conteúdo e tendências em redes sociais, dentre outros \citep{uysal2012novel, gupta2018text}.

Alguns exemplos desses algoritmos são: \textit{Decision Tree} ou Árvore de decisão, \textit{Support Vector Machine} ou Máquina de Vetores de Suporte, \textit{Neural Network} ou Redes Neurais Artificiais e \textit{K-Nearest Neighbors} ou K-vizinhos mais próximos. 

\cite{uysal2012novel} propuseram uma nova metodologia para extração de características em bases de dados textuais. E com o objetivo de avaliar a eficácia do novo método, executaram uma sequência de testes em quatro diferentes bases de dados, contendo estas diferentes características, com diferentes medidas de avaliação e por fim comparou o novo método com os algoritmos citados.

\begin{itemize}
    \item Árvore de decisão: um classificador não supervisionado e linear, baseado na estrutura matemática de árvore, onde cada decisão é chamada de um nó nessa árvore. O processo de decisão é feito através do percorrimento a partir do nó raiz, até o nó folha da árvore. O uso mais comum desse método é para decisões binárias, ou seja, decisões que possuem apenas duas alternativas, verdadeira ou falsa.
    \item \textit{Support Vector Machine}:
    \item \textit{Neural Network}
    \item \textit{K-Nearest Neighbors}
\end{itemize}

\section{Chamadas emergenciais falsas: Problemas e Soluções}
Chamadas emergenciais falsas são consideradas um fardo para os serviços emergenciais. Tais chamadas são responsáveis por perda de tempo, esforço, energia, recursos, enfim um bônus para o Estado.

Segundo \cite{waseem2010prank}, em uma análise feita sobre os dados das chamadas emergenciais recebidas entre outubro de 2004 e Maio de 2010, em Punjab, Paquistão, realçou que $97,85\%$ foram identificadas como falsas, onde, mais precisamente, $91,5\%$ destas foram identificadas como trote (brincadeiras ou insultos ao atendente); $7,27\%$ por busca de informações não relacionadas ao serviço; $1,5\%$ de chamadas por engano (discagem de número errado); e $0,14\%$ de chamadas onde o solicitante forjou a necessidade de um atendimento emergencial. 
Diante do exposto, ações foram efetivadas: a implantação de um sistema de monitoramento de chamadas e um sistema de bloqueio de chamadas, no qual após três ou quatro chamadas falsas ou trotes originadas do mesmo número, o bloqueio da linha telefônica seria realizado, impossibilitando assim, a efetuação de chamadas temporariamente.
Entretanto, as medidas propostas não obtiveram grande êxito, uma vez que não houve redução significativa da incidência dessas chamadas indevidas.

\cite{rashford2010optimizing}, destacam a importância da otimização e do uso apropriado do sistema de chamadas emergenciais, combatendo os falsos chamados. Os autores realçam o desgaste financeiro que tal prática resulta na sociedade, retirando recursos que poderiam ser aproveitados em pacientes que realmente necessitam desse atendimento e transporte. Como medida preventiva, campanhas públicas de conscientização foram elaboradas, mas novamente não obtiveram os resultados esperados. Resultando assim, a introdução de penas mais severas, na legislação vigente da Austrália,  para quem realizasse falsos chamados, podendo ser multado em até \$ 10.000 ou 1 ano de prisão. 
Ainda sobre a descrição dos serviços de ambulância no referido país, é relatado o uso do Sistema de Prioridades para Despacho Médico (do inglês: \textit{Medical Priority Dispatch System} - MPDS). Neste sistema, o atendente é orientado a efetuar uma série de perguntas ao solicitante, com o objetivo de obter uma melhor descrição da emergência. Após essa triagem primária, a situação é então classificada em um código alfanumérico previamente definido, composto por: um número, entre 1 a 33 (Tabela \ref{tab:mpdsCodes}), que descreve o principal sintoma identificado; uma letra, \textit{Alpha}, \textit{Beta}, \textit{Charlie}, \textit{Echo} ou \textit{Omega}, que classifica a prioridade da emergência; e por fim, um outro número, que procura especificar ainda mais o problema.
Este sistema, bem como outros sistemas similares, foram elaborados com o objetivo de alcançar um padrão comum, reduzindo o critério subjetivo e pessoal dos atendentes sobre a situação do paciente. Há dados que denotam a eficiência destes sistemas, onde é possível obter mais informações sobre o estado do paciente, aumentando assim a chance de sobrevivência, prestando um atendimento mais específico e eficiente \citep{gray2008ampds}.

Os estudiosos relatam ainda que as chamadas menos prioritárias devem ainda passar por uma segunda triagem, sendo esta, por médicos especialistas, com o objetivo de um diagnóstico mais preciso. Sendo possível, mudar o prognóstico anterior, aumentando o nível de prioridade ou até mesmo não efetivando o despacho da ambulância ao local e sim o desencadeamento de medidas alternativas para o atendimento \citep{marks2002emergency, gray2008ampds}.

\begin{table}[ht!]
\caption{Lista de códigos traduzidos do MPDS.}
\label{tab:mpdsCodes}
\centering
\begin{tabular}{| c | L{4.5cm} | c | L{4.5cm} |}
\hline
\multicolumn{4}{|c|}{Códigos MPDS} \\
\hline
1 & Dor abdominal & 18 & Dor de cabeça \\
2 & Reação alérgica & 19 & Problema no coração \\
3 & Mordida por animal & 20 & Exposição a calor ou frio \\
4 & Assalto& 21 & Hemorragia \\
5 & Dor nas costas& 22 & Acidente industrial \\
6 & Dificuldade para respirar& 23 & Overdose \\
7 & Queimadura & 24 & Gravidez \\
8 & Exposição a elemento químico perigoso & 25 & Problema psiquiátrico \\
9 & Ataque cardíaco & 26 & Doença \\
10 & Dor no peito & 27 & Esfaqueamento ou tiro \\
11 & Asfixia & 28 & Acidente vascular \\
12 & Convulsão & 29 & Acidente de trânsito \\
13 & Diabete & 30 & Traumatismo \\
14 & Afogamento & 31 & Indivíduo inconsciente \\
15 & Eletrocução & 32 & Sintoma não identificado \\
16 & Problema ocular & 33 & Cuidado paliativo \\
17 & Queda & & \\
\hline
\end{tabular}
{\fontsize{11pt}{\baselineskip}\selectfont
}
\end{table}

Devido ao alto índice de trotes recebidos por serviços de telefonia mundial, tal como o serviço de Operador de Chamadas Internacionais (do inglês: International Operator Direct Calling - IODC), que têm como função conectar um turista em um viagem internacional, através de uma chamada, com um atendente de seu país natal. \cite{kuroiwa2004automatic} propuseram um sistema de identificação automática de trotes e chamadas falsas, baseado na tecnologia de reconhecimento de fala. Esse sistema foi desenvolvido para operar com base em ligações para o Japão, onde é verificado se a pessoa que deseja utilizar do serviço, compreende bem o idioma japonês. O teste é validado solicitando o usuário a repetir uma palavra em japonês, esperando um determinado tempo pela pronuncia. Caso o usuário pronuncie corretamente a palavra, pressupõem-se que o mesmo conhece o dialeto da região o qual está telefonando, portanto as chances de ser uma ligação real são altas, sendo assim, o usuário é rapidamente redirecionado para o atendente. Caso contrário, uma segunda e última tentativa de pronuncia é solicitada ao usuário. Se o mesmo errar ou não pronunciar a palavra, um aviso é dado e a ligação é encerrada. Como resultados da metodologia proposta, os autores informaram que depois de analisar cerca de $100.000$ chamadas, um total de $9489$ foram identificadas como chamadas reais, apresentando uma taxa de acerto de $97\%$ e um taxa de rejeição de chamadas falsas de $93\%$. O volume de trotes eram tão alto que houve dias na semana em que a quantidade ultrapassava $6.000$ de ligações rejeitadas.

\section{Privacidade dos dados e ética}
Das bases de dados disponíveis, os conteúdos apresentam-se limitados, ou seja, com poucas informações, o que pode dificultar as pesquisas sobre o tema. Tal limitação denota  especialmente no aspecto à privacidade dos dados, uma vez que, por serem sigilosos, não podem ser compartilhados de forma a expor os indivíduos envolvidos.

A garantia da preservação do segredo das informações, além de uma obrigação legal contida na maioria dos Códigos de Ética profissional e também no Código Penal Brasileiro \citep{cp340}, é um dever \textit{prima facie} de todos e para com todos. Além do conteúdo sigiloso presente, faz-se necessário a efetivação de uma etapa referente a Ética:

\renewenvironment{quote}[1][1em]
  {\begin{adjustwidth}{#1}{0}}
  {\end{adjustwidth}}
\begin{quote}[4cm]
\begin{singlespace}
{\footnotesize  
Nas atividades de pesquisa, muitas vezes são utilizados dados constantes em prontuários e bases de dados. Essa utilização deve ser resguardada e permitida apenas para projetos previamente aprovados por um Comitê de Ética em Pesquisa, desde que plenamente descaracterizada a identificação do paciente, inclusive quanto as suas iniciais e registro hospitalar. Mesmo nas publicações científicas não deve ser possível identificar os pacientes através de fotografias ou outras imagens. Em caso de necessidade imperiosa, isto será permitido apenas com o consentimento, por escrito, dos mesmos o que possui amparo na própria Constituição Federal, em seu Art. 5º, item X. \citep{francisconi1998aspectos}.
}
\end{singlespace}
\end{quote}

Entretanto como apontado por \cite{francisconi1998aspectos}, existem situações que claramente constituem exceções à preservação de segredos, devido ao risco de vida associado ou ao benefício social que pode ser obtido. Se tratando de uma área com poucas pesquisas realizadas, inexistem resultados que comprovem a possível eficácia de um sistema que solucione a identificação correta de chamadas falsas em serviços de emergência. E consequentemente, os gestores que administram tais dados se sentem menos propensos a contribuir com pesquisas, por desconhecer o potencial que tais ferramentas poderiam oferecer ao processo.

Em suma, é fundamental compreender o respeito e o devido cuidado que todas as informações dos usuários desses serviços emergenciais merecem, como também o desenvolvimento de estratégias para tratar-las de forma eticamente adequada.