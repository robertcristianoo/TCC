\chapter{Introdução}
Com o objetivo de ampliar, regulamentar e aumentar a eficiência dos serviços de urgência no Brasil, o Ministério da Saúde publicou duas portarias em 2003, a GM/MS nº 1863 e a GM/MS nº 1864 que instituíram, respectivamente, a Política Nacional de Atenção às Urgências e o componente pré-hospitalar móvel, por meio da implantação do Serviço de Atendimento Móvel de Urgência (SAMU), disponível através do número telefônico $192$, como também os serviços de urgência associados, em todo o território nacional \citep{p1863, p1864}. A implantação desses projetos resultou em impactos positivos para toda a população \citep{VIEIRA2008, MACHADO2011, MINAYO2008}. 

Contudo, também surgiram novos problemas e desafios. Um dos principais problemas encontrados são as chamadas falsas ou trotes, no qual o indivíduo entra em contato com a unidade de atendimento e relata uma situação inexistente, que se não identificada pelo atendente, pode resultar em deslocamentos desnecessários de unidades e de recursos ao local do chamado. Outra situação comum são as chamadas com categorização não emergencial, em que a população procura por orientações, tais como números de outros serviços, locais e endereços. 

O grande problema dessas ligações é que elas ocupam a linha emergencial, consomem recursos, geram transtornos e custos para os serviços de emergência. Este é um fator preocupante que, pode acarretar a escassez desses recursos em chamadas não emergenciais. Caso haja outra pessoa que realmente necessite de atendimento, ela poderá não receber os auxílios necessários a tempo e ir a óbito, em um grave acidente de trânsito, por exemplo.

Além desses riscos, uma mobilização indevida de recursos, tais como pessoas, viaturas, ambulâncias, carros de combate a incêndio e, principalmente, de tempo, geram também prejuízos financeiros ao Estado. E esses recursos que poderiam ser investidos em outros setores, como educação, infraestrutura, segurança ou aperfeiçoamento dos próprios serviços de emergência, por exemplo, acabam tendo que ser remanejados. Gastos estes que poderiam ser parcialmente evitados, se os atendentes puderem identificar melhor as chamadas falsas.

O ato do trote aos serviços de emergência é um crime previsto no Código Penal brasileiro. Segundo o Art. nº 340 do Código Penal \citep{cp340}, "Provocar a ação de autoridade, comunicando-lhe a ocorrência de crime ou de contravenção que sabe não se ter verificado", quando identificado o autor, o mesmo pode ser detido por um período de um a seis meses ou multado. Com uma análise cautelosa ao artigo, percebe-se que o mesmo não abrange a comunicação falsa de situações de emergência que motivem o acionamento do Serviço de Atendimento Móvel de Urgência (SAMU) ou corpo de bombeiros, e sim somente nos casos em que há relato de infrações penais. Como relatado por \cite{peixoto2015combate}, existem outros artigos e projetos de leis que poderiam ajudar a punir tal ato, mas devido a sua natureza, talvez não sejam a melhor opção para o problema. 

Embora que, grande parte dos autores de chamados falsos são crianças e adolescentes, o problema se encontra em chamadas falsas iniciadas por adultos, pois o conteúdo dessas mensagens tendem a serem mais próximas de uma ocorrência real. Dificultando assim, a identificação de ser uma chamada falsa pelo atendente.

Após consultas na literatura disponível, não foram encontrados muitos resultados que visam identificar tais chamadas emergenciais falsas. O que leva a crer que é uma área pouco visada e que existam outros fatores que influenciem o baixo investimento em pesquisas e soluções. Um desses fatores é que o conteúdo dessas ligações são sensíveis a privacidade e sujeitos a análise por um conselho de ética \citep{francisconi1998aspectos}. Sendo essa, uma barreira e um processo burocrático, porém necessário para o desencadeamento de pesquisas. Além disso, os gestores (responsáveis pelos dados) podem se sentir menos dispostos a disponibilizar dados anônimos para pesquisas devido a falta de divulgação de resultados positivos.

Quando se trabalha com metodologias para a indústria, como por exemplo, na identificação e classificação de falhas em equipamentos, há um grande números de dados abertos de funcionamento real \citep{phmsociety2009datasets, nasa2007datasets, uci2017datasets}, que favorece a proposição de soluções para as mais variadas demandas da área. Entretanto, existem poucas bases de dados públicas disponíveis para realizações de experimentos e prova de metodologias, que seriam capazes de solucionar o problema de identificação de chamadas falsas. E ainda dentre tais bases, a maioria está disponível no idioma inglês, e os dados que são disponibilizados são rasos, carecendo de detalhes e proporcionando pouco espaço para uma exploração aprofundada.

Portanto, a presente proposta não busca realizar apenas um experimento de classificação. O trabalho visa realizar um estudo de avaliação aprofundado e crítico sobre quais procedimentos técnicos devem ser realizados, qual o menor conjunto de informações a serem coletadas das chamadas, e quais questões de privacidade devem ser levantadas para que seja viável a construção de sistemas de suporte a decisão do gestor e do atendente nos serviços emergenciais.

\section{Motivação}
Segundo um estudo apontado pela Consultoria Legislativa do Senado, estima-se que o custo gerado por trotes aos serviços de emergência no Brasil, tais como o Serviço de Atendimento Móvel de Urgência (SAMU), Corpo de Bombeiros (CB) e Polícia Militar (PM), chega a R\$ $1$ bilhão por ano. Esse levantamento foi apurado pela PM do estado do Amapá, no qual eles avaliaram que a cada deslocamento para um atendimento emergencial incompleto, gera um custo aproximado de R\$ $500,00$ \citep{globo2014trotes, peixoto2015combate}.

Devido à proporção do problema, os atendentes do Centro Integrado de Operações de Segurança Pública (CIOSP) do governo de Mato Grosso, responsáveis estes pelo atendimento e despacho das chamadas emergenciais, receberam uma capacitação para identificar suspeitas de trote. Mesmo que não seja possível bloquear todas ligações indevidas, a triagem ajuda a amenizar os prejuízos \citep{govMatoGrosso2016}. 

Ainda assim, nas pesquisas realizadas preliminarmente, não foram localizadas, no meio acadêmico, propostas de sistemas computacionais, que auxiliassem a resolução do problema. Diante desse cenário, que possui poucas pesquisas, visibilidade, resultados e também o não uso de técnicas computacionais para auxiliar os serviços de emergência, surge então, a possibilidade do levantamento de requisitos que seriam necessários para a construção de sistemas computacionais que auxiliem na identificação correta de chamadas falsas nos serviços de emergência. Busca-se ainda, motivar as autoridades envolvidas no processo, para que, invistam no desenvolvimento dessas ferramentas.

\section{Objetivos}
Deseja-se mostrar a viabilidade de produzir sistemas capazes de identificação de chamadas falsas, como também produzir documentos e argumentos para que os gestores desses serviços possam avaliar a possibilidade de liberação do acesso aos dados reais e completos das chamadas, sem prejuízo ou violação de privacidade e com o retorno de uma ferramenta de apoio a decisão aos operadores. Com a conclusão da avaliação, também será possível elencar quais dados podem ser suficientes ou necessários para a identificação.

Como objetivos específicos alcançam-se:

\begin{itemize}
    \item Pesquisar o uso de técnicas computacionais que possam auxiliar os serviços de atendimento de emergência;
    \item Apontar como métodos de classificação de dados, junto ao processamento de linguagem natural possa viabilizar a construção de sistemas que auxiliem a identificação de chamadas falsas;
    \item Motivar, com resultados positivos, as autoridades responsáveis, para aumentar o investimento no desenvolvimento de ferramentas;
\end{itemize}

\section{Estrutura do trabalho}
Os demais itens que compõem este trabalho, estão organizados na seguinte estrutura:

\begin{itemize}
    \item No capítulo 2 são apresentados conceitos básicos relacionados a classificação de dados, processamento de linguagem natural e classificação de texto. Além disso, dois algoritmos de classificação de texto são descritos, as redes neurais artificiais e o k-vizinhos mais próximos.
    \item No capítulo 3 é apresentado uma revisão na literatura, abordando os problemas gerados por chamadas falsas, como também tentativas e soluções propostas para reduzir o impacto consequente. Em seguida apresenta-se trabalhos da literatura sobre a classificação de texto, com o objetivo de compreender suas vantagens, capacidades e eficiência, quando aplicados em contextos diferentes ou semelhantes ao deste trabalho. Por fim, um tópico abordando sobre privacidade e ética que estão sujeitos os dados dessas chamadas emergenciais, bem como, as possíveis vantagens que seriam obtidas caso esses dados fossem de mais fácil acesso.
    \item No capítulo 4 é apresentado a metodologia proposta para alcançar os objetivos especificados, como também uma descrição da base de dado que será utilizada e o cronograma de atividades esperadas para a próxima etapa.
\end{itemize}